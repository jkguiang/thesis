% Your fancy acks here. Keep in mind you need to ack each paper you
% use. See the examples here. In addition, each chapter ack needs to
% be repeated at the end of the relevant chapter.
\begin{acknowledgements}
I have long wondered what I would write in this section. 
In times of crisis, I would draft the words in my mind; I would find strength in the people whose the shoulders I stand upon. 
With the date of my defense now rapidly approaching, it is humbling to finally enumerate that long list of people, and it is truly a challenge to express my sincere gratitude to them in mere sentences. 
Nevertheless, please find below my best attempt at thanking those people who made this dissertation, these last five years, and my nearly 27 years on Earth more full and meaningful than I think any one person is owed.

Foremost, I owe boundless thanks to my incredible family. 
Although it appears at the beginning of this text, this section took the longest to write and was finished well after this section was otherwise complete, so deeply do I care for the words and names that follow. 
% Dad
% Mom
% Christina
% Grandma & Grandpa
% Rosemary

% TODO: make this read better, instead of 'I thank' at the beginning, thank at the start of paragraph, then list people?
Next, I am deeply indebted to the many mentors I have had in my relatively short academic career. 
These people served as my guides, teachers, and councilors. 
I must first thank Prof. Frank W\"urthwein, my advisor, for believing in me, supporting me, and inspiring me to be bold---alongside a great deal of lessons in physics, statistics, and more. 
I thank also Prof. Avi Yagil, my co-advisor, for challenging, uplifting, and educating me. 
I am ever indebted to Prof. Claudio Campagnari for taking me in and completely altering the course of my life. 
Claudio gave me purpose during what was, probably unbeknownst to him and many others, a very difficult time for me personally---thank you, Claudio; without you, this dissertation simply would not exist. 
I also owe many thanks to Prof. Philip Chang for hours of conversation via Zoom and over 120,000 Skype messages (not counting the exclusive DPU Octagon), for even more hours of physics and C++ lessons, for helping me navigate my academic career, and for his nearly endless patience. 
I thank Prof. Indara Suarez for welcoming me into particle physics, for pushing me to meet my true potential, and for supporting me throughout my entire career in academic research. 
I thank Dr. Nick Amin for teaching me everything I know and for guiding me, once again, as I enter a new stage of my life. 
I thank also Dr. Bennett Marsh and Dr. Sam May for lessons in Python, machine learning, and mentorship---I am a better leader, person, and scientist thanks to Nick, Bennett, and Sam. 
Finally, I thank all of the members, past and present, of the illustrious ``Surf n' Turf'' (SNT) empire, whose dominion spans the continental United States of America and beyond. 
It has been a priviledge to serve as your resident Data Librarian, and I am deeply grateful to each of you, though listing all of your names here would surely inflate the length of this document well beyond the patience of its reviewers. 
Therefore, my thanks are extended, but not limited to the following SNT'ers: Prof. Frank Golf, Dr. Daniel Spitzbart, (soon to be Dr.) Aashay Arora, Dr. Slava Krutelyov, and former SNT'er Jerry Ling.

I have also had the priviledge of working alongside a number of incredible research engineers whom I would like to mention here. 
In particular, I owe thanks to Diego Davila for patiently teaching some computer science to a budding physicist and for being a great mentor and friend. 
Thanks also to Igor Sfiligoi, Dr. Fabio Andrijauskas, and Terrence Martin for the support, much-needed lunch breaks, and even gym lessons! 

Lastly, I would like to acknowledge my incredible friends who have kept me sane and made me a better person. 
I thank Joey Incandela for being my physics brother (and best man), for the many game nights, and for his compassion and spirit. 
I thank Gabe Hernandez for being my ex-physics brother (and ex-roommate), for the silly inside jokes, and for his guidance and encouragement. 
I thank Grady Kestler for being my partner in crime, for Sunday Night Games, and for his camaraderie and insight. 

Chapter~\ref{ch:lhc_cms} describes the Large Hadron Collider and the CMS Experiment. 
The figures used in this chapter are obtained from the following references: XYZ, XYZ, XYZ. % TODO: fixme!

Chapter~\ref{ch:vbswh} describes a search for the production of a W and Higgs boson via vector boson scattering, using data recorded by the CMS experiment from 2016 to 2018. 
It is a partial reproduction of the paper ``Study of $\WH$ production through vector boson scattering with extraction of the relative sign of the $\PW$ and $\PZ$ couplings to the Higgs boson'', Phys. Lett. B. 2024, XYZ (2024). % TODO: fixme!

Chapter~\ref{ch:vbsvvh} describes a search for the production of a Higgs boson and two vector bosons via vector boson scattering, using data recorded by the CMS experiment from 2016 to 2018. 
It is a partial reproduction of a paper being prepared for submission for publication of the material. 

Chapter~\ref{ch:lst} describes improvements to the Line Segment Tracking algorithm using machine learning. 
It is a partial reproduction of the paper ``Improving tracking algorithms with machine learning: a case for line-segment tracking at the High Luminosity LHC'', in the proceedings of Connecting the Dots 2023, arXiv:2403.13166

Chapter~\ref{ch:cyber} describes improvements to the Line Segment Tracking algorithm using machine learning. 
It is a partial reproduction of the paper ``Integrating End-to-End Exascale SDN into the LHC Data Distribution Cyberinfrastructure'', in the proceedings of PEARC 2023, doi:10.1145/3491418.3535134

\end{acknowledgements}
