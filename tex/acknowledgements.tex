\begin{acknowledgements}
I have long wondered what I would write in this section. 
In times of crisis, I would draft the words in my mind, seeking strength in the people whose the shoulders I stand upon. 
With the date of my defense now rapidly approaching, it is humbling to finally enumerate that long list of people, and it is truly a challenge to express my sincere gratitude to them in mere sentences. 
Nevertheless, please find below my best attempt at thanking those people who made this dissertation, my career, and my nearly 27 years on Earth more full and meaningful than I think any one person is owed.

Foremost, I owe boundless thanks to my incredible family. 
First, to Mom and Dad, thank you for raising me in a home full of love and imagination and for all of your work and sacrifice upon which my life, career, and now this dissertation has been built upon. 
And to Christina, my big sister: thank you for your strength and courage, for all of our memories growing up together (except, perhaps, those that ended with my injury), and for inspiring me to be bold. 
I am also fortunate to have spent a great deal of time with my grandparents, John and Tsuneko Hashiguchi---thank you, grandma and grandpa, for teaching me to be good and honest and for showing me how to perservere. 
Amidst all of my ``found'' family---family by bond, not blood---I must, most of all, thank my uncle, John Sanders, for putting up with his briny nickname and for always being there for me---from our regular calls at UCSB, to helping me decide on a graduate program, to helping me decide on staying in that program, to helping me take the next step in my career. 
Finally, to my wife Rosemary, who has brightened my life in every way: thank you for your kidness and compassion, for everything that you are, I love you more than anything. 
I hope that today and on every day after, I can make you all proud.

Next, I am deeply grateful for the many mentors I have had in my relatively short academic career, organized roughly by institution. 
First, I thank Prof. Frank W\"urthwein, my advisor, who believed in me, guided me through all of my achievements, and flew me around the world---thank you, Frank, for everything. 
And to Prof. Avi Yagil, who served as my co-advisor: thank you for challenging, uplifting, and educating me. 
I also owe many thanks to Prof. Philip Chang who, in many ways, was also my co-advisor: for hours of conversation via Zoom and over 130\,000 Skype messages, for even more hours of physics and C++ lessons, and for your nearly endless patience, thank you, Philip. 

I am ever indebted to Prof. Claudio Campagnari for taking me in and completely altering the course of my life. 
Claudio gave me purpose during what was, probably unbeknownst to him and many others, a very difficult time for me personally---thank you, Claudio; without you, this dissertation simply would not exist. 
Claudio's former students and postdocs were my first and most formative mentors. 
Thanks, in particular, to Dr. Nick Amin for teaching me everything I know. 
Thanks also Dr. Bennett Marsh and Dr. Sam May for lessons in Python and machine learning---I am a better leader, person, and scientist thanks to Nick, Bennett, and Sam. 
And to Prof. Indara Suarez, who welcomed me into particle physics and pushed me to meet my true potential: thank you for your mentorship, the high-octane tour of Boston in your Mustang, and your constant support. 

I must thank all of the members, past and present, of the illustrious ``Surf n' Turf'' (SNT) empire, whose dominion stretches far across the western world. 
It has been a priviledge to serve as your Data Librarian, and I am deeply grateful to each of you, though listing all of your names here would surely inflate the length of this document well beyond the patience of its reviewers. 
Therefore, my thanks are extended, but not limited to the following SNT'ers: Prof. Frank Golf, Dr. Daniel Spitzbart, (soon to be Dr.) Aashay Arora, Dr. Slava Krutelyov, and former SNT'er Jerry Ling.

I have also had the priviledge of working alongside a number of incredible research engineers whom I would like to mention here. 
In particular, I owe thanks to Diego Davila for patiently teaching some computer science to a budding physicist and for being a great mentor and friend. 
Thanks also to Igor Sfiligoi, Dr. Fabio Andrijauskas, and Terrence Martin for the support, much-needed lunch breaks, and even gym lessons! 

Lastly, I would like to acknowledge my incredible friends who have kept me sane and made me a better person. 
To Joey Incandela, my physics brother (and best man): thank you for the many game nights (despite the 3-hour time difference), for keeping me sane, and for your compassion and spirit. 
To Gabe Hernandez, my ex-physics brother (and ex-roommate): thank you for the silly inside jokes, for your constant guidance and encouragement, and for accompanying me to the base of Mt. San Jacinto, where I finished writing this dissertation amidst pizza, cookies, and the pines. 
To Grady Kestler, last by not least, my partner in crime: thank you for Sunday Night Games, for getting me through Stat. Mech., and for your camaraderie and insight. 

Chapter~\ref{ch:lhc_cms} describes the Large Hadron Collider (LHC) and the CMS Experiment. 
The figures used in this chapter are materials produced by or for CERN and were released by CERN for informational use under CERN copyright~\cite{CERNCopyright}. 

Chapter~\ref{ch:vbswh} describes a search for the production of a W and Higgs boson via vector boson scattering, using data recorded by the CMS experiment from 2016 to 2018. 
It is a partial reproduction of a paper being prepared for submission for publication of the material. 
The material that this chapter is based on, however, has already been made public: \url{https://cds.cern.ch/record/2882655}.

Chapter~\ref{ch:vbsvvh} describes a search for the production of a Higgs boson and two vector bosons via vector boson scattering, using data recorded by the CMS experiment from 2016 to 2018. 
It is a partial reproduction of a paper being prepared for submission for publication of the material. 

Chapter~\ref{ch:lst} describes improvements to the Line Segment Tracking algorithm using machine learning. 
It is a partial reproduction of the paper ``Improving tracking algorithms with machine learning: a case for line-segment tracking at the High Luminosity LHC'', in the proceedings of Connecting the Dots 2023, arXiv:2403.13166

Chapter~\ref{ch:cyber} describes the development of software-defined networking capabilities within the LHC data distribution cyberinfrastructure. 
It is a partial reproduction of the paper ``Integrating End-to-End Exascale SDN into the LHC Data Distribution Cyberinfrastructure'', in the proceedings of PEARC 2023, doi:10.1145/3491418.3535134

\end{acknowledgements}
