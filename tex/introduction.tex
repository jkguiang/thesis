\begin{dissertationintroduction}
\begin{aquote}{Carl Sagan, Cosmos: A Personal Voyage, 1980}
    We are a way for the cosmos to know itself.
\end{aquote}

\section*{Infinity in the palm of your hand}
The history of particle physics began with a simple question: what is everything made of? 
From the Greek philosopher Democritus~\cite{Taylor1419554}, to the Hindu scholar Kan\=ada (or, ``atom-eater''), and probably long before them, people have mused about what that most fundamental unit of everything might be. % citation needed for Kanada
For a vast majority of human history, however, all theories were equally immaterial. 
Nevertheless, as science progressed, faltered, corrected, and progressed again (and again, and again), patterns began to emerge, presented here as what David Griffiths calls a ``folk history'' for brevety, with inspiration taken heavily from the first chapter of his revered textbook~\cite{GriffithsParticle}.  

In 1808, the chemist John Dalton published the first\footnotemark{} atomic theory~\cite{Dalton1805, Dalton1808}, but the existence of the atom itself could not be directly verified. 
\footnotetext{
    Dalton's now legendary claim to this was hotly contested by his contemporary, William Higgins, although without much credence~\cite{HigginsClaim}.
}
However, nearly a century later, physicists including Henri Becquerel, Marie Curie, and J.J. Thompson found that certain elements and electronic apparatuses emitted different kinds of invisible radiation~\cite{Becquerel, Curie, RADVANYI2017544, Thompson}. 
In the decades following, they further realized that this radiation consisted of tiny particles\footnotemark{}, and Ernest Rutherford directed rays of these particles at thin sheets of metals, probing the subatomic scale for the first time. 
Based on these seminal experiments, Rutherford made a groudbreaking observation and two key deductions: % citations needed for Rutherford
\footnotetext{
    Namely photons (X-rays or $\gamma$ rays), electrons ($\beta$ rays), and helium ions (He$^{2+}$ or $\alpha$ rays). 
}
\begin{enumerate}
    \item The atom exists\footnotemark{} and has a very small ``nucleus'' of positive charge.
    \item The nucleus is made of positively charged ``protons.''
    \item The nucleus is also made of ``neutrons'' which somehow help keep it together.
\end{enumerate}
\footnotetext{
    It should be noted that Jean Perrin separately confirmed the existence of molecules, and thus reality of atoms, by proving experimentally that the random walk of a microscopic object suspended in liquid could only be due to collisions with the molecules of that liquid~\cite{Perrin}---a phenomenon first observed by the great botanist Robert Brown~\cite{Brown} (i.e. Brownian motion) and rigorously explained by Albert Einstein in one of his first papers~\cite{EinsteinBrownian}.
}
Unbeknownst to them, Rutherford, Curie, and the other atomic pioneers had stumbled upon the surface of an even greater truth than Democritus, Kan\=ada, or anyone else could have imagined. 

Albert Einstein, Max Planck, Niels Bohr, and others had begun writing down corrections to ``classical'' physics at the turn of the 19\ts{th} century in order to better describe unexplained phenomena observed in the lab~\cite{EinsteinPhotoelectric, Planck, Bohr}. 
By time the Rutherford's neutrons had been discovered by James Chadwick in 1932~\cite{Chadwick1932}, a cadre of theorists, including many of the most well-known physicists in history\footnotemark{}, had formed around Quantum Mechanics.
This ``new physics'' extended the early results of Einstein, Planck, and Bohr into a more complete theory describing the subatomic world, then thought to only include protons, neutrons, and electrons. 
\footnotetext{Werner Heisenberg, Erwin Schr\"odinger, and Robert Oppenheimer to name a few.}
Over the next decades, a slew of new particles were discovered: 
anti-matter was produced in the lab; % citation needed
electrons were joined by muons and neutrinos in the ``lepton'' family; % citations needed
an enormous crowd of more massive particles (baryons and mesons which together comprise the families of hadrons) akin to protons and neutrons were observed. % select citations needed
While some of these particles were already thought to exist, experimental results had, momentarily, outpaced theory. 
Theorists were developing new ideas on how to classify or relate different particles, but the ever-growing ``particle zoo'' was otherwise in dissarray. 
However, as is a consistent pattern in history, science progressed, faltered, corrected, and progressed again (and again, and again), and patterns began to emerge. 

\section*{A glorious victory parade}
It began with Murray Gell-Mann, who, among others, saw a pattern in the deluge of new particles being discovered. 
He arranged the numerous hadrons in geometric patterns---octogons, triangles, and so forth---according to their properties~\cite{Gell-Mann:1961omu}, and found that his unusual diagrams held predictive power: some of his shapes were missing vertices, which were later found to be particles exactly matching the expectations from Gell-Mann's theory~\cite{PhysRevLett.12.204}. 
With confidence in Gell-Mann's so-called ``Eightfold Way'' established, a natural question arose: what is the significance of these patterns? 
Gell-Mann himself, and, separately, George Zweig, arrived at the answer: hadrons are not themselves fundamental particles; instead, they are composed of truly fundamental particles called ``quarks''~\cite{Lichtenberg:784713}, and the Eightfold Way diagrams corresponded to different configurations of quarks and properties of those configurations.
Similar to Rutherford's experiments 50 years prior, wherein beams of particles were used to probe the structure of the atom, the reality of quarks could be probed by firing beams of electrons at protons, revealing their inner structure~\cite{PhysRevLett.23.930, PhysRevLett.23.935}. 
Though these experiments showed only that the proton had three constituent ``partons,'' they provided initial evidence for the existence of quarks.
And so the field progressed, with families of particles filling out into neat organizations: 
the quarks became a family of six, divided into three generations (up/down, charm/strange, top/bottom); % select citations needed
the leptons, with the addition of the tau and its neutrino, satisfyingly also totaled six particles divided into three generations ($\Pe/\PGne$, $\PGm/\PGnGm$, $\PGt/\PGnGt$); % citation for tau needed
the photon was joined by the gluon, \PW, \PZ, and Higgs boson~\cite{ATLASdisc, CMSdisc}, forming the family of bosons. % select citations needed
At the same time, a beautiful, yet deeply perplexing, quantum field theory was developing, describing how each of these particles interact. 
In the words of the great Sydney Coleman~\cite{Coleman}:
\begin{quote}
    This was a great time [1966 to 1979] to be a high-energy theorist, the period of the famous triumph of quantum field theory. 
    And what a triumph it was, in the old sense of the word: a glorious victory parade, full of wonderful things brought back from far places to make the spectator gasp with awe and laugh with joy. 
\end{quote}
So began the Standard Model of particle physics: with groundbreaking discoveries and a scientific ``victory parade,'' which together still represent the largest fraction of Nobel Prizes by field~\cite{ParticleNobels}. 

\section*{There and back again}
The Standard Model has, despite remaining largely unchanged since the 1980s, withstood a tremendous amount of scrutiny across decades of intense study and many billions of dollars of scientific instrumentation. % citation or LHC xsec plot
Meanwhile, the now long-known errors and inconsistencies in the Standard Model have been, much to the community's dismay, equally durable. % citations needed
Particle physics has therefore reached a familiar inflection point, where we are left in anticipation of the next measurement that confirms an unusual approach or the next surprising discovery that leads the theory in an entirely new direction. 
Described in this document are the efforts of one graduate student, along with the invaluable support of his many colleagues, to contribute to this extraordinary enterprise. 
Together, we shall recount his journey, taking the perspective of the ``experimentalist.'' 
That is, very few formulas, axioms, and derivations---the language of theorists---are to be found in this dissertation. 
Instead, we will find tables of measurements, histograms, and some statistics. 
We will take general direction from the theory, then construct experiments to realize it in nature. 

This work is based on the proton-proton collision data collected by the Compact Muon Solenoid (CMS), a multi-billion dollar particle detector situated on the beamline of the Large Hadron Collider (LHC)---the most powerful particle collider and the single largest scientific instrument ever built, at the time of writing. 
At the risk of losing the reader who is not committed to reading the many pages that lie beyond this introduction, no new physics is documented here.
Instead, this humble subission, amongst over one thousand publications from the CMS Collaboration alone, represents a small contribution to the rich history documented above, demonstrating the continued success of a theory that dares to describe the \textit{entire universe}.
\end{dissertationintroduction}
