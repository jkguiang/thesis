% Optional Introduction
\begin{dissertationintroduction}
The history of particle physics began with a simple question: what is everything made of? 
The concepts of atoms and particles, however, were not the obvious answers, though they are rooted in ancient thoughts: from the Greek philosopher Democritus~\cite{Taylor1419554}, to the Hindu scholar Kan\=ada (or, ``atom-eater''). % citation needed for Kanada
Instead, for a vast majority of human history, all theories were equally immaterial. 
As late as 1808, when John Dalton published the first\footnotemark{} atomic theory~\cite{Dalton1805, Dalton1808}, the existence of the atom itself could not be directly verified. 
\footnotetext{
    Dalton's now legendary claim to this was hotly contested by his contemporary, William Higgins, although without much credence~\cite{HigginsClaim}.
}
Nevertheless, as science progressed, faltered, corrected, and progressed again (and again, and again), patterns began to emerge, presented here as what David Griffiths calls a ``folk history'' for brevety, with inspiration taken heavily from his revered textbook~\cite{Griffiths}.  

In the burgeoning field of late-nineteenth-century physics, Henri Becquerel, Marie Curie, J.J. Thompson, and others found that certain elements and electronic apparatuses emitted different kinds of invisible radiation~\cite{Becquerel, Curie, RADVANYI2017544, Thompson}. 
In the decades following, they further realized that this radiation consisted of tiny particles\footnotemark{}, and Ernest Rutherford directed rays of these particles at thin sheets of metals, probing the subatomic scale for the first time. 
Based on these seminal experiments, Rutherford made a groudbreaking observation and two key deductions: % citations needed for Rutherford
\footnotetext{
    Namely photons (X-rays or $\gamma$ rays), electrons ($\beta$ rays), and helium ions (He$^{2+}$ or $\alpha$ rays). 
}
\begin{enumerate}
    \item The atom exists\footnotemark{} and has a very small ``nucleus'' of positive charge.
    \item The nucleus is made of positively charged ``protons.''
    \item The nucleus is also made of ``neutrons'' which somehow help keep it together.
\end{enumerate}
\footnotetext{
    It should be noted that Jean Perrin separately confirmed the existence of molecules, and thus reality of atoms, by proving experimentally that the random walk of a microscopic object suspended in liquid could only be due to collisions with the molecules of that liquid~\cite{Perrin} --- a phenomenon first observed by the great botanist Robert Brown~\cite{Brown} (i.e. Brownian motion) and rigorously explained by Albert Einstein in one of his first papers~\cite{EinsteinBrownian}.
}
Thus, the atomic age began in the lab. 
That is, the works cited so far contain little-to-no mathematical formulas or postulates --- the language of theorists --- but rather the tables of measurements and detailed observations of experimentalists. 
While there was serious theoretical motivation for the existence of atoms\footnotemark{}, the structure of the atom itself had not been correctly described. 
\footnotetext{
    Daniel Bernoulli, for example, had explained temperature as the kinetic motion of molecules decades before Dalton made his observations, and James Maxwell, Ludwig Boltzmann, Josiah Gibbs, and others had formulated the statistical mechnics of gases and other collections of microscopic objects well before Rutherford's experiments.
}
These atomic pioneers, however, had stumbled upon the surface of a greater truth than Democritus, Kan\=ada, or anyone else could have imagined. 
By time the Rutherford's neutrons had been discovered by James Chadwick in 1932~\cite{Chadwick1932}, the formulation of a New Physics was well underway, marking the beginning of a long (and yet incomplete) journey towards the true description of the entire universe. 

While Rutherford and others were tinkering with radiation in the lab, Albert Einstein, Max Planck, Niels Bohr, and others began writing down corrections to ``classical'' physics in order to better describe unexplained phenomena observed in the lab~\cite{EinsteinPhotoelectric, Planck, Bohr}. 
By the 1930s, a cadre of theorists, including many of the most well-known physicists in history\footnotemark{}, had formed around Quantum Mechanics, which extended the early results of Einstein, Planck, and Bohr into a more complete theory describing the subatomic world. 
\footnotetext{Werner Heisenberg, Erwin Schr\"odinger, and Jon von Neumann to name a few.}
Over the next decades, a slew of new particles where either discovered in the lab or theorized to exist, with strong support from other observations: 
anti-matter was proposed, then produced in the lab; 
electrons were joined by muons, neutrinos, and their anti-particles in the ``lepton'' family; 
whole families of more massive particles (baryons and mesons which together comprise the families of hadrons), akin to protons and neutrons, were observed. 
Experimental results had again, momentarily, outpaced theory. 
Although theorists were developing new ideas on how to classify or relate different particles, the ever-growing collection of particles was otherwise in dissarray. 
However, as is a consistent pattern in history, science progressed, faltered, corrected, and progressed again (and again, and again), and patterns began to emerge. 

It began with Murray Gell-Mann, who, among others, saw a pattern in the deluge of new particles discovered. 
He arranged them in geometric patterns --- octogons, triangls, and so forth --- according to their properties~\cite{Gell-Mann:1961omu}, and found that his unusual ``Eightfold Way'' held predictive power: some of his shapes were missing vertices, which were later found to be particles exactly matching the expected properties expected by Gell-Mann's theory~\cite{PhysRevLett.12.204}. 
With confidence in Gell-Mann's Eightfold Way established, the natural question was: what is the significance of these patterns? 
Gell-Mann himself, and, separately, George Zweig, arrived at the answer: hadrons are not themselves fundamental particles like electrons, photons, and the like; instead, they are composed of truly fundamental particles called ``quarks''~\cite{Lichtenberg:784713}.
Similar to Rutherford's experiments 50 years prior, the reality of these particles could be probed by firing beams of electrons at protons, revealing its inner structure~\cite{PhysRevLett.23.930, PhysRevLett.23.935}. 
Though these experiments showed only that the proton had three constituent ``partons,'' they provided strong initial evidence for the existence of quarks, which was later proven more concretely.
\end{dissertationintroduction}
