\chapter{Elucidating the Higgs boson at CMS}
% In this chapter, a simple, yet powerful insight is derived from the statistical analysis of proton-proton data collected by the CMS Experiment. 
% This analysis serves as a blueprint for the following chapter, in which a more involved, but thematically similar, analysis is documented. 
% Moreover, its structure reveals how CMS data is used to make statements about the Standard Model in general. 
% First, a ``signal'' process is defined, motivated by some exciting new theory or by the simple fact that it has never been considered before. 
% However, the signal process is often incredibly rare compared to the sheer amount of uninteresting junk (``background'' processes) produced at the LHC. 
% The signal is therefore carefully studied for features that distinguish it from the background. 
% Most likely, some background processes have similar, or even identical, features. 
% A strategy is therefore derived to optimally select the signal with as little background contamination as possible. 
% This optimization is performed using petabytes of simulated proton-proton collisions (``events''), 
% Finally, some statistical method is employed to make a precise statement on the precision of the final measurement. 

\section{Chasing new physics with the Higgs boson}
The stage is set: over a century of particle physics has yielded a beautiful, but incomplete theory of everything, the Standard Model, and a grand coalition of nations have built the largest and most complex scientific instrument in human history, the LHC, to test it. 
The most recent triumph came in 2012, when the Higgs boson was discovered. 
In the years following its discovery, however, the LHC experiments have measured many of its properties to great precision and found no significant deviations from SM predictions~\cite{NatureHiggsCMS2022, NatureHiggsATLAS2022}. 
Nevertheless, the confounding mysteries still surrounding the Higgs boson at the time of writing suggest that there must be some beyond Standard Model (BSM) physics that is not yet understood. 
Moreover, given the existential importance of the Higgs boson, this kind of new physics would have profound implications towards a better understanding of the past, present, and future of the entire universe. 

There are many educated guesses, called theories, aimed at addressing these open questions around the Higgs boson. 
Some guess at the existence of yet-undiscovered particles that also interact with the Higgs boson\footnotemark{}. 
\footnotetext{This is not an unlikely guess: dark matter is known only due to its gravitational pull, which implies that it has mass, which further suggests it obtains that mass through the Higgs mechanism.}
Experimentalists can also search for new physics indirectly by making precise measurements of SM predictions; any significant deviation from the prediction would poke another hole in the Standard Model or even confirm a prediction of a new theory. 
The physics analyses described in this document both follow the latter strategy.

\subsection{The $k$-framework}
One commonly used framework used to quantify these deviations from the SM is the so-called $k$-framework~\cite{KFrame}, which introduces modifiers $\kappa_X$ to the Higgs boson couplings to some particle $X$:
\begin{equation}
    \kappa_X = \frac{\text{modifed coupling value}}{\text{SM coupling value}}.
\end{equation}
While there are myriad theoretical nuances to the statement above, it is sufficient to state the obvious: $\kappa_X = 1$ represents the SM scenario and significant deviations from 1 represent BSM scenarios. 

\section{Simulation}

\section{Data}

\section{Event selection}
\subsection{Optimization}
\subsection{Background estimation}

\section{Statistical interpretation}
