\chapter{Doing Particle Physics with CMS}
\begin{aquote}{Richard Feynman, The Feynman Lectures, 1963}
The principle of science, the definition, almost, is the following: The test of all knowledge is experiment. 
Experiment is the sole judge of scientific ``truth.''
\end{aquote}

\section{Looking for new physics}
The stage is set: over a century of particle physics has yielded a beautiful, but incomplete theory of everything, the Standard Model, and a grand coalition of nations have built the largest and most complex scientific instrument in human history, the LHC, to test it. 
The most recent triumph came in 2012, when the Higgs boson was discovered. 
We will focus on questions around the Higgs boson, although this is just one of many ongoing efforts at the LHC.

In the years following the discovery of the Higgs boson, the LHC experiments have measured many of its properties to great precision and found no significant deviations from SM predictions~\cite{NatureHiggsCMS2022, NatureHiggsATLAS2022}. 
Nevertheless, the confounding mysteries still surrounding the Higgs boson at the time of writing suggest that there must be some beyond Standard Model (BSM) physics that is not yet understood. 
Moreover, given the existential importance of the Higgs boson, this kind of new physics would have profound implications towards a better understanding of the past, present, and future of the entire universe. 
There are many educated guesses, called theories, aimed at addressing these open questions around the Higgs boson. 
Some guess at the existence of yet-undiscovered particles that also interact with the Higgs boson\footnotemark{}. 
\footnotetext{This is not an unlikely guess: dark matter is known to have mass, and it may obtain that mass through the Higgs mechanism.}
Experimentalists can also search for new physics indirectly by making precise measurements of SM predictions; any significant deviation from the prediction would poke another hole in the Standard Model or even confirm a prediction of a new theory. 
The physics analyses described in this document both follow the latter strategy.

\subsection{The $k$-framework}
One commonly used framework used to quantify these deviations from the SM is the so-called $k$-framework~\cite{KFrame}, which introduces modifiers $\kappa_X$ to the Higgs boson couplings to some particle $X$:
\begin{equation}
    \kappa_X = \frac{\text{modifed coupling value}}{\text{SM coupling value}}.
\end{equation}
While there are myriad theoretical nuances to the statement above, it is sufficient to state the obvious: $\kappa_X = 1$ represents the SM scenario and significant deviations from 1 represent BSM scenarios. 

\section{Simulation}

\section{Reconstruction}
\subsection{Leptons}
\subsection{Jets}
\subsubsection{Jets originating from VBS quarks}
\subsubsection{Jets originating from b quarks}

\section{Event selection}
\subsection{Triggers}
\subsection{Signal regions}
\subsection{Control regions}
\subsection{Background estimation}

\section{Statistical interpretation}
