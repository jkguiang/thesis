% Stupid vita goes next
\begin{vita}
\noindent
\begin{cv}{}
\begin{cvlist}{}
\item[2019] Bachelor of Science, University of California Santa Barbara
\item[2019--2020] Teaching Assistant, University of California San Diego
\item[2020--2024] Research Assistant, University of California San Diego
\item[2023] Master of Science, University of California San Diego
\item[2024] Doctor of Philosophy, University of California San Diego
\end{cvlist}
\end{cv}

% This puts in the PUBLICATIONS header. Note that it appears inside
% the vita environment. It is optional.
\publications
\noindent Guiang, J. et al. (2023). 
\textit{Improving tracking algorithms with machine learning: a case for line-segment tracking at the High Luminosity LHC}
In 8th International Connecting The Dots Workshop. arXiv:2403.13166.
\newline
\newline
\noindent Aashay Arora, Jonathan Guiang, Diego Davila, Frank Würthwein, Justas Balcas, \& Harvey Newman (2023). 
\textit{400Gbps benchmark of XRootD HTTP-TPC.}
In 26th International Conference on Computing in High Energy \& Nuclear Physics. arXiv:2312.1258.
\newline
\newline
\noindent W\"urthwein, F., Guiang, J. et al. (2022). 
\textit{Managed Network Services for Exascale Data Movement Across Large Global Scientific Collaborations.}
In 2022 4th Annual Workshop on Extreme-scale Experiment-in-the-Loop Computing (pp. 16-19). IEEE Computer Society.
\newline
\newline
\noindent Guiang, J. et al. (2022). 
\textit{Integrating End-to-End Exascale SDN into the LHC Data Distribution Cyberinfrastructure.}
In Practice and Experience in Advanced Research Computing. Association for Computing Machinery.
\newline
\newline
\noindent Fajardo, E., ..., Guiang, J. et al. (2020). 
\textit{Moving the California distributed CMS XCache from bare metal into containers using Kubernetes.}
In 24th International Conference on Computing in High Energy \& Nuclear Physics. EPJ Web of Conferences.
\newline
\newline
\noindent Ball, A., ..., Guiang, J. et al. (2020). 
\textit{Search for millicharged particles in proton-proton collisions at $\sqrt{s} = 13\TeV$.}
Physical Review D, 102(3).
\end{vita}
